
% LaTeX fragment containing introductory section for PDF report document

\section{Preface}

\subsection{Resources}

This QC (quality control) report accompanies two further QC resources:
\begin{enumerate}
\item Summary of sample metadata and QC results: pipeline\_summary.csv. A comma-separated text file (.csv) comprising one line per sample, with associated annotation and QC status. The .csv file can be viewed in any spreadsheet application.
\item Detailed QC datafiles and companion plots within the release.  These appear in the `plate heatmaps' and `supplementary' subdirectories.
% TODO Reference to detailed QC documentation
\end{enumerate}

\subsection{Metrics}

Six QC metrics are used to determine sample quality and handling effects.  All are summarised in this report.

Each metric requires appropriate threshold criteria to include or exclude samples.  For the \textbf{duplicate} metric the threshold is fixed, and for the \textbf{gender} metric it is calculated independently for each pipeline analysis.  Other thresholds are configured before the pipeline is run; see Section \ref{sec:metric-summary} for the values used.

\subsubsection*{Metric descriptions}

\begin{enumerate}
\item \textbf{Identity:} In addition to Illumina genotyping, samples may be typed on an alternate platform such as Sequenom or Fluidigm. Identity of calls between platforms is checked.  The check requires a minimum number of sites to be shared between platforms (default is 8). If fewer than the minimum are available, the identity check is not applied and \emph{all} samples receive a pass. Otherwise, only samples for which concordance is greater than a chosen threshold will pass.

See Section \ref{sec:metric-summary} for threshold values in this pipeline run.

\item \textbf{Duplicate:}  Calls for each sample are compared with all other samples.  Pairs of samples with a SNP concordance $>$ 0.98 are flagged as duplicates (unless otherwise stated).  The sample with the lowest call rate of the pair is recommended for exclusion.  Note, pairs of samples with a similarity score $>$ 0.70 are flagged also but not recommended for exclusion.

\item \textbf{Gender:} Supplied gender is compared to the inferred gender. Gender is inferred using heterozygosity on sample chrX, which is the proportion of chrX SNPs called AB. A Gaussian mixture model is used to find adaptive thresholds $M_{max}$ and $F_{min}$, respectively the maximum male and minimum female heterozygosity on chrX.  Samples above $M_{max}$ and below $F_{min}$ are considered to be of ambiguous gender.  Any sample whose inferred gender is ambiguous, or conflicts with the supplied gender, is recommended for exclusion.

See Section \ref{sec:metric-summary} for threshold values in this pipeline run.

\item \textbf{Call rate:} Sample call rate is calculated as the proportion of SNPs called successfully for that sample.  Samples with call rate below a given threshold are excluded (unless otherwise stated). 

See Section \ref{sec:metric-summary} for threshold values in this pipeline run.

\item \textbf{Autosomal heterozygosity:}  Sample heterozygosity is calculated as the proportion of all SNPs with an AB call.  Samples with an autosomal heterozygosity score more than a given number of standard deviations away from the mean are recommended for exclusion (unless otherwise stated).  

See Section \ref{sec:metric-summary} for threshold values in this pipeline run.

\item \textbf{Magnitude of intensity:} Samples are recommended for exclusion where the normalized magnitude of intensity signal in both channels falls below the given threshold (unless otherwise stated). As of pipeline version 1.2.0, this metric is not applied to GenoSNP because of differing intensity normalization.

See Section \ref{sec:metric-summary} for threshold values in this pipeline run.
\end{enumerate}
